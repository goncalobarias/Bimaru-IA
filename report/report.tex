\documentclass[12pt,a4paper]{article}
\usepackage[legalpaper, portrait, margin=2cm]{geometry}
\usepackage{fancyhdr}
\usepackage{amsmath}
\usepackage{amssymb}
\usepackage{graphicx}
\usepackage{wrapfig}
\usepackage{blindtext}
\usepackage{hyperref}
\usepackage{tikz}
\usepackage{adjustbox}
\usepackage{siunitx}
\usepackage{booktabs}
\usepackage{svg}
\usepackage{subfig}
\usepackage{caption}
\usepackage{indentfirst}
\usepackage[none]{hyphenat}

\graphicspath{ {./} }
\hypersetup{
  colorlinks=true,
  linkcolor=blue,
  filecolor=magenta,
  urlcolor=blue,
  citecolor=blue,
  pdftitle={Guião Relatório IA - 2022/2023 LEIC-A},
  pdfpagemode=FullScreen,
}

\pagestyle{fancy}
\fancyhf{}
\rhead{Grupo \textbf{al010}}
\lhead{Guião Relatório IA 2022/2023 LEIC-A}
\cfoot{Gonçalo Bárias (103124) e Raquel Braunschweig (102624)}

\definecolor{pastel-green}{HTML}{CBE896}
\definecolor{pastel-yellow}{HTML}{FEE440}

\renewcommand{\footrulewidth}{0.2pt}

\renewcommand{\labelitemii}{$\circ$}
\renewcommand{\labelitemiii}{$\diamond$}
\newcommand{\op}{\operatorname}

\begin{document}
\section{Introdução}
Bom dia, vimos aqui apresentar a nossa solução para um solucionador do jogo \textit{Bimaru}.
A nossa solução formaliza o jogo \textit{Bimaru} como um \textbf{Problema de Satisfação de Restrições}.

\section{Modelação do Problema e Estratégia de Solução}
Cada \textbf{estado} do tabuleiro é representado por uma grelha quadrada de dimensões $10$ x $10$ em que
o estado inicial é lido do \textit{standard input}. Mantemos ainda contadores do número de peças de barco e
número de peças de água por linha e coluna e do número de barcos colocados no tabuleiro.
Assim, as variáveis que tivemos em consideração foi cada posição do tabuleiro, sendo os domínios de todas igual
às letras que representam cada tipo de peça, tal como foi sugerido na representação externa do tabuleiro
apresentado no enunciado. Para as posições por preencher decidimos usar um ponto de interrogação.

% Explicar as restrições do problema(nenhum barco a tocar noutro, número de barcos tem de ser exatamente igual ao número para cada tamanho,
% linhas e colunas têm de ter igual número de peças que as fornecidas em rows_num(i) e cols_num(i)).
% Explicar como fazemos o problema obedecer às restrições na nossa solução.
% Explicar que as verificamos sempre que efetuamos uma alteração no tabuleiro, sendo que isto permite reduzir drasticamente o fator
% de ramificação, diminuindo o custo de memória, e evitando tempo de computação desperdiçado.

% No goal_test mencionamos que como temos como invariante as restrições sempre todas verificadas, apenas precisamos de tempo constante
% para ver se todos os barcos estão colocados.
% Explicar como efetuamos inferência no nosso problema de CSP com o reduce_board e como é que isso reduz ainda mais o fator de ramificação (prunning),
% completando barcos antes de ter de criar sucessores com ele colocado.

% Explicar como usamos procura no nosso CSP junto com a inferência para chegar à solução (o que ê uma action e como obtemos actions,
% o que ê o resultado de uma action, o que ê o goal_test, falar de forward checking, backtracking sempre que encontramos tabuleiro inválido, etc.).
% Mencionar que seguimos a heurística de maior grau ao escolher sempre o barco de maior tamanho possível, pois isso atribui valores
% à maior quantidade de variáveis, escolhendo assim mais variáveis que estão envolvidas em restrições (nomeadamente restrições do número por linha/coluna).
% No forward checking verifica-se a consistência da variável alterada para com as variáveis adjacentes na árvore de restrições.

\section{Avaliação Experimental}
% Falar em concreto das procuras usadas, tal como a nossa heurística.
% Mostrar gráficos das procuras e explicar o porquê do DFS ser bom o suficiente e até melhor que
% as procuras informadas (a nossa heurística acaba por não ser boa).
% Contrastar BFS com DFS.

\end{document}
