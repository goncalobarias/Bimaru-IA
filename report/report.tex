\documentclass[12pt,a4paper]{article}
\usepackage[legalpaper, portrait, margin=2cm]{geometry}
\usepackage{fancyhdr}
\usepackage{hyperref}
\usepackage{wasysym}
\usepackage{indentfirst}

\hypersetup{
  colorlinks=true,
  linkcolor=blue,
  filecolor=magenta,
  urlcolor=blue,
  citecolor=blue,
  pdftitle={Guião Relatório IA - 2022/2023 LEIC-A},
  pdfpagemode=FullScreen,
}

\pagestyle{fancy}
\fancyhf{}
\rfoot{Grupo \textbf{al010}}
\lfoot{Gonçalo Bárias (103124) e Raquel Braunschweig (102624)}

\renewcommand{\headrulewidth}{0pt}
\renewcommand{\footrulewidth}{0.2pt}

\begin{document}

\section*{\center{Guião do Relatório de IA}}
\vspace*{10pt}

Bom dia, vimos aqui apresentar um solucionador do jogo \textit{Bimaru}.
A nossa solução formaliza o jogo \textit{Bimaru} como um \textbf{Problema de Satisfação de Restrições}.

Cada \textbf{estado} do tabuleiro é representado por uma grelha quadrada de dimensões $10$ x $10$ em que
o estado inicial é lido do \textit{standard input}. Mantemos ainda contadores do número de peças de barco e
número de peças de água por linha e coluna e do número de barcos colocados no tabuleiro.
Assim, as variáveis que tivemos em consideração foi cada posição do tabuleiro, sendo os domínios de todas, igual
às letras que representam cada peça, tal como foi sugerido na representação externa do tabuleiro
apresentado no enunciado. Para as variáveis por atribuir decidimos usar um ponto de interrogação (\texttt{?}) e para
as variáveis que apenas têm peças de barco no domínio decidimos usar a letra \texttt{x}.

O objetivo do jogo é preencher totalmente o tabuleiro com peças, de tal forma que o número de peças de barco em cada linha ou coluna tem de ser
precisamente igual ao número fornecido como input para essa linha ou coluna, têm também de existir quatro submarinos, três contratorpedeiros,
dois cruzadores e um couraçado dispostos horizontal ou verticalmente, em que nenhum deles pode se encontrar adjacente a nenhum outro, nem
mesmo na diagonal.
Assim, sempre que é feita uma atribuição a uma das variáveis, a nossa solução apenas a aceita se não quebrar nenhuma das restrições
mencionadas. Tendo em consideração que para estados intermédios apenas não se pode exceder o número de barcos e os números de peças
por linha e coluna, mas é possível ter valores inferiores para esses contadores.

Uma \textbf{ação} é representada por um 4-tuplo, em que as primeiras duas entradas representam a posição de um extremo do barco, a terceira entrada representa o
tamanho do barco e a última entrada representa a orientação do barco através da peça da ponta. O \textbf{resultado} de uma ação é atribuir as variáveis
do tabuleiro com peças do barco numa cópia do estado, de modo a colocar esse barco no tabuleiro.
Considerou-se a Heurística do Maior Grau, ou seja, quando se geram os sucessores de um estado damos sempre prioridade ao barco do maior tamanho possível
que obedece ao número de barcos no tabuleiro, pois conseguimos atribuir valores concretos a mais variáveis que, por sua vez estão envolvidas
num maior número de restrições do que no caso de se considerar um barco menor.
Estas otimizações permitem reduzir o fator de ramificação da árvore de procura, diminuindo o custo de memória e tempo de computação desnecessário.
Escolheu-se utilizar \textit{forward checking} para unir a inferência com os algoritmos de procura e assim após uma ação, é verificada a consistência
no grafo de restrições e no caso de alguma variável ficar com domínio vazio é efetuado \textit{backtracking}.

A procura usada foi uma procura cega, pois tendo em consideração que o tamanho do tabuleiro está limitado, a árvore nunca é grande
ao ponto das procuras informadas ajudarem de uma forma significativa. De entre as procuras cegas foi escolhida a \texttt{DFS}, pois a \texttt{BFS}
expande a árvore quase toda, já que o nó solução está no último nível da árvore, na maior parte dos casos.

Como nenhuma das restrições é quebrada, então o \textbf{teste objetivo} pode ser feito em tempo constante, pois apenas temos de ver se todos
os barcos foram utilizados.
A solução tenta sempre reduzir o tabuleiro o máximo possível em cada estado, ou seja, nas linhas e colunas em que o número de peças por colocar
é igual ao número de variáveis por atribuir mais o número de variáveis atribuídas com peças de barco ou em que o número de peças por colocar é zero,
restringem-se as variáveis restantes dessas linhas e colunas com os únicos valores dos seus domínios que obedecem a essa restrição, de modo a manter a consistência.
Assim, a nossa solução propaga sempre as restrições, limitando o domínio das variáveis. Neste exemplo, quando é atribuída uma peça de topo a uma variável,
conseguimos restringir o domínio das variáveis adjacentes no tabuleiro.
Temos aqui outro exemplo em que o nosso programa consegue limitar o domínio da variável para apenas \texttt{t}, pois o seu domínio era apenas
peças de barco e como se encontra rodeado por variáveis todas atribuídas, consegue-se inferir a sua atribuição.
Isto possibilita a descoberta de novos barcos antes de ter de procurar posições para ele, o que reduz de forma significativa o fator de ramificação.

E assim terminamos a nossa apresentação, espero que tenham gostado. \smiley{}

\end{document}
